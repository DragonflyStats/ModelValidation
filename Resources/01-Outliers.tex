\documentclass[residuals.tex]{subfiles}
\begin{document}
\newpage
\section{Outliers and Influential Observations}
\begin{quote}
	"Outliers are sample values that cause surprise in relation to the majority of the sample" (W.N. Venables and B.D. Ripley. 2002. Modern applied statistics with S. New York: Springer, p.119).
\end{quote}

Crucially, surprise is in the mind of the beholder and is dependent on some explicit model of the data. 

Importantly, Normality is only an assumption:There may be another model under which the outlier is not surprising at all, say if the data really are lognormal or 
gamma rather than normal. 
\newpage
\subsection{Outliers}

Data points that diverge in a big way from the overall pattern are referred to as ``outliers".\\ 
In the case of Simple Linear Regression, there are four ways that a data point might be considered an outlier.
%---------------------------------------------------------%

\begin{itemize}
	\item It could have an extreme X value compared to other data points.
	\item It could have an extreme Y value compared to other data points.
	\item It could have extreme X and Y values.
	\item It might be distant from the rest of the data, even without extreme X or Y values.
\end{itemize}
\newpage
%---------------------------------------------------------%
\begin{itemize}
\item After a regression line has been computed for a group of data, a point which lies far from the line 
(and thus has a large residual value) is known as an outlier. 
Such points may represent erroneous data, or may indicate a poorly fitting regression line. 

\item If a point lies far from the other data in the horizontal direction, it is known as an \textit{\textbf{influential observation}}. 
The reason for this distinction is that these points have may have a significant impact on the slope of the regression line.
\end{itemize}
\newpage
\subsection*{\texttt{outlierTest()}}
Suppose we have a two fitted models and we would like to see if there are any outliers. 

For this purpose, we can use \texttt{outlierTest()} from \texttt{library(car)} in R. 

%Currently facing a problem in interpreting the results.


\begin{framed}
\begin{verbatim}
library(car)
outlierTest(fit1)   

**Result:**
    rstudent unadjusted p-value  Bonferonni p
21    -4.12            4.39e-05        0.0209
15     -4.08            5.39e-05        0.0257

outlierTest(fit2)   

**Result:**
No Studentized residuals with Bonferonni p < 0.05
Largest |rstudent|:
    rstudent unadjusted p-value      Bonferonni p
177    -2.52             0.0119                NA
\end{verbatim}
\end{framed}

The row numbers ( here : 21, 15 and 177) indicate the outlier points in the data.



\end{document}
